\documentclass[b5paper, 12pt, oneside]{book}
\usepackage{verbatim}
\usepackage{amsmath}
\usepackage{textcomp}
\usepackage[official]{eurosym}
\usepackage{CJKutf8}
\usepackage{amsfonts}
\usepackage{graphicx}
\usepackage{makeidx}
\usepackage[colorlinks = true,linkcolor = blue, citecolor = magenta]{hyperref}
\begin{document}
\title{\Huge Tao Te Ching}
\author{\textsc{Lao Tzu} \\ \\ \normalsize \textit{Translated by Stephen Mitchell}}
\date{}
\maketitle
\mainmatter
\chapter*{1}
The tao that can be told \\
is not the eternal Tao. \\
The name that can be named\\
is not the eternal Name. \\
\\
The unnamable is the eternally real.\\
Naming is the origin\\
of all particular things.\\
\\
Free from desire, you realise the mystery.\\
Caught in desire, you see only the manifestations.\\
\\
Yet mystery and manifestations\\
arise from the same source.\\
This source is called darkness.\\
\\
Darkness within darkness.\\
The gateway to all understanding.\\

\chapter*{2}
When people see some things as beautiful,\\
other things become ugly.\\
When people see some things as good,\\
other things become bad.\\
\\
Being and non-being create each other.\\
Difficult and easy support each other.\\
Long and short define each other.\\
High and low depend on each other.\\
Before and after follow each other.\\
\\
Therefore the Master\\
acts without doing anything\\
and teaches without saying anything.\\
Things arise and she lets them come;\\
things disappear and she lets them go.\\
She has but doesn't possess,\\
acts but dosen't expect.\\
When her work is done, she forgets it.\\
That is why it lasts forever.\\

\chapter*{3}
If you overesteem great men,\\
people become powerless.\\
If you overvalue possessions,\\
people begin to steal.\\
The Master leads\\
by emptying people's minds\\
and filling their cores,\\
by weakening their ambition\\
and toughening their resolve.\\
He helps people lose everything\\
they know, everything they desire,\\
and creates confusion\\
in those who think that they know.\\
\\
Practise not-doing,\\
and everything will fall into place.\\

\chapter*{4}
The Tao is like a well;\\
used but never used up.\\
It is like the eternal void;\\
filled with infinite possibilities.\\
\\
It is hidden but always present.\\
I don't know who gave birth to it.\\
It is older than God.\\

\chapter*{5}
The Tao doesn't take sides;\\
it gives birth to both good and evil.\\
The Master doesn't take sides;\\
she welcomes both saints and sinners.\\
\\
The Tao is like a bellows;\\
it it empty yet infinitely capable.\\
The more you use it, the more it produces;\\
the more you talk of it, the less you understand.\\
\\
Hold on to the centre.\\



\end{document}
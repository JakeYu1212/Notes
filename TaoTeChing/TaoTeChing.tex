\documentclass[a4paper, 12pt]{book}
\usepackage{verbatim}
\usepackage{amsmath}
\usepackage{textcomp}
\usepackage[official]{eurosym}
\usepackage{CJKutf8}
\usepackage{amsfonts}
\usepackage{graphicx}
\usepackage{makeidx}
\usepackage[colorlinks = true,linkcolor = blue, citecolor = magenta]{hyperref}
\begin{document}
\title{Tao Te Ching}
\author{Lao Tzu}
\date{}
\maketitle
\mainmatter
\chapter{}
The tao that can be told \\
is not the eternal Tao. \\
The name that can be named\\
is not the eternal Name. \\
\\
The unnamable is the eternally real.\\
Naming is the origin\\
of all particular things.\\
\\
Free from desire, you realise the mystery.\\
Caught in desire, you see only the manifestations.\\
\\
Yet mystery and manifestations\\
arise from the same source.\\
This source is called darkness.\\
\\
Darkness within darkness.\\
The gateway to all understanding.\\

\chapter{}

\end{document}
\documentclass[b5paper, 12pt, oneside]{book}
\usepackage{verbatim}
\usepackage{amsmath}
\usepackage{textcomp}
\usepackage[official]{eurosym}
\usepackage{CJKutf8}
\usepackage{amsfonts}
\usepackage{graphicx}
\usepackage{makeidx}
\usepackage[colorlinks = true,linkcolor = blue, citecolor = magenta]{hyperref}
\usepackage{fancyhdr}
\usepackage{xcolor}
\begin{document}
\title{\Huge Tao Te Ching}
\author{\textsc{Lao Tzu} \\ \\ \normalsize \textit{Translated by Stephen Mitchell}}

\date{} % take out the date on the title page

\maketitle
\clearpage
\phantom{s}
\thispagestyle{empty}

\mainmatter

\pagestyle{fancyplain} %使用fancyplain风格
\fancyhf{} %清除所有页眉页脚

\chapter*{1}
The tao that can be told \\
is not the eternal Tao. \\
The name that can be named\\
is not the eternal Name. \\
\\
The unnamable is the eternally real.\\
Naming is the origin\\
of all particular things.\\
\\
Free from desire, you realise the mystery.\\
Caught in desire, you see only the manifestations.\\
\\
Yet mystery and manifestations\\
arise from the same source.\\
This source is called darkness.\\
\\
Darkness within darkness.\\
The gateway to all understanding.\\

\chapter*{2}
When people see some things as beautiful,\\
other things become ugly.\\
When people see some things as good,\\
other things become bad.\\
\\
Being and non-being create each other.\\
Difficult and easy support each other.\\
Long and short define each other.\\
High and low depend on each other.\\
Before and after follow each other.\\
\\
Therefore the Master\\
acts without doing anything\\
and teaches without saying anything.\\
Things arise and she lets them come;\\
things disappear and she lets them go.\\
She has but doesn't possess,\\
acts but dosen't expect.\\
When her work is done, she forgets it.\\
That is why it lasts forever.\\

\chapter*{3}
If you overesteem great men,\\
people become powerless.\\
If you overvalue possessions,\\
people begin to steal.\\
The Master leads\\
by emptying people's minds\\
and filling their cores,\\
by weakening their ambition\\
and toughening their resolve.\\
He helps people lose everything\\
they know, everything they desire,\\
and creates confusion\\
in those who think that they know.\\
\\
Practise not-doing,\\
and everything will fall into place.\\

\chapter*{4}
The Tao is like a well;\\
used but never used up.\\
It is like the eternal void;\\
filled with infinite possibilities.\\
\\
It is hidden but always present.\\
I don't know who gave birth to it.\\
It is older than God.\\

\chapter*{5}
The Tao doesn't take sides;\\
it gives birth to both good and evil.\\
The Master doesn't take sides;\\
she welcomes both saints and sinners.\\
\\
The Tao is like a bellows;\\
it it empty yet infinitely capable.\\
The more you use it, the more it produces;\\
the more you talk of it, the less you understand.\\
\\
Hold on to the centre.\\

\chapter*{6}
The Tao is called the Great Mother:\\
empty yet inexhaustible,\\
it gives birth to infinite worlds.\\
\\
It is always present within you.\\
You can use it any way you want.\\

\chapter*{7}
The Tao is infinite, eternal.\\
Why is it eternal?\\
It was never born:\\
thus it can never be die.\\
It has no desires for itself:\\
thus it is present for all beings.\\
\\
The Master stays behind:\\
that is why she is ahead.\\
She is detached from all things:\\
that is why she is one with them.\\
Because she has let go of herself,\\
she is perfectly fulfilled.\\

\chapter*{8}
The supreme good is like water,\\
which nourishes all things without trying to.\\
It is content with the low places that people disdain.\\
\\
In dwelling, live close to the ground.\\
In thinking, keep to the simple.\\
In conflict, be fair and generous.\\
In governing, don't try to control.\\
In work, do what you enjoy.\\
In family life, be completely present.\\
\\
WHen you are content to be simply yourself\\
and don't compare or compete,\\
everybody will respect you.

\chapter*{9}
Fill your bowl to the brim\\
and it will spill.\\
Keep sharpening your knife\\
and it will blunt.\\
Chase after money and security\\
and your heart will never unclench.\\
Care about people's approval\\
and you will be their prisoner.\\
\\
Do your work, then step back.\\
The only path to serenity.\\

\chapter*{10}
Can you coax your mind from its wandering\\
and keep to the roiginal oneness?\\
Can you let your body become\\
supple as a newborn child's?\\
Can you cleanse your inner vision\\
until you see nothing but the light?\\
Can you love people and lead them\\
without imposing your will?\\
Can you deal with the most vital matters\\
by letting events take their course?\\
Can you step back from your own mind\\
and thus understand all things?\\
\\
Giving birth and nourishing,\\
having without possessing,\\
acting with no expectations,\\
leading and not tring to control:\\
this is the supreme virtue.\\

\chapter*{11}
We join spokes together in a wheel,\\
but it is the centre hole\\
that makes the wagon move.\\
\\
We shape clay into a pot,\\
but it is the emptiness inside\\
that holds whatever we want.\\
\\
We hammer wood for a house,\\
but it is the inner space\\
that makes it livable.\\
\\
We work with being,\\
but non-being is what we use.\\

\chapter*{12}
Colours blind the eye.\\
Sounds deafen the ear.\\
Flavours numb the taste.\\
Thoughts weaken the mind.\\
Desires wither the heart.\\
\\
The Master observes the world\\
but trusts his inner vision.\\
He allows things to come and go.\\
His heart is open as the sky.\\

\chapter*{13}
Success is as dangerous as failure.\\
Hope is as hollow as fear.\\
\\
What does it mean that success is as dangerous as failure?\\
Whether you go up the ladder or down it,\\
your position is shaky.\\
When you stand with your two feet on the ground,\\
you will always keep your balance.\\
\\
What does it mean the hope is as hollow as fear?\\
Hope and fear are both phantoms\\
that arise from thinking of the self.\\
When we don't see the self as self,\\
what do we have to fear?\\
\\
See the world as your self.\\
Have faith in the way things are.\\
Love the world as your self;\\
then you can care for all things.\\

\chapter*{14}
Look, and it can't be seen.\\
Listen, and it can't be heard.\\
Reach, and it can't be grasped.\\
\\
Above, it isn't bright.\\
Below, it isn't dark.\\
Seamless, unnamable,\\
it returns to the realm of nothing.\\
Form that includes all forms,\\
image without an image,\\
subtle, beyond all conception.\\
\\
Approach it and there is no beginning;\\
follow it and there is no end.\\
You can't know it, but you can be it,\\
at ease in your own life.\\
Just realise where you come from:\\
this is the essence of wisdom.\\

\chapter*{15}
The ancient Masters were profound and subtle.\\
Their wisdom was unfathomable.\\
There is no way to describe it;\\
all we can describe is their appearance.\\
\\
They were careful\\
as someone crossing an iced-over stream.\\
Alert as a warrior in enemy territory.\\
Courteous as a guest.\\
Fluid as melting ice.\\
Shapable as a block of wood.\\
Receptive as a valley.\\
Clear as a glass of water.
\\
Do you have the patience to wait\\
till your mud settles and the water is clear?\\
Can you remain unmoving\\
till the right action arises by itself?\\
\\
The Master doesn't seek fulfilment.\\
Not seeking, not expecting,\\
she is present, and can welcome all things.\\

\chapter*{16}
Empty your mind of all thoughts.\\
Let your heart be at peace.\\
Watch the turmoil of beings,\\
but contemplate their return.\\
\\
Each separate being in the universe\\
returns to the common source.\\
Returning to the source is serenity.\\
\\
If you don't realise the source,\\
you stumble in confusion and sorrow.\\
When you realise where you come from,\\
you naturally become tolerant,\\
disinterested, amused,\\
kind-hearted as a grandmother,\\
dignified as a king.\\
Immersed in the wonder of the Tao,\\
you can deal with whatever life brings you,\\
and when death comes, you are ready.\\

\chapter*{17}
When the Master governs, the people\\
are hardly aware that he exists.\\
Next best is a leader who is loved.\\
Next, one who is feared.\\
The worst is one who is despised.\\
\\
If you don't trust the people,\\
you make them untrustworthy.\\
\\
The Master doesn't talk, he acts.\\
When his work is done,\\
the people say,``Amazing:\\
we did it, all by ourselves!"\\

\chapter*{18}
When the great Tao is forgotten,\\
goodness and piety appear.\\
When the body's intelligence declines,\\
cleverness and knowledge step forth.\\
When there is no peace in the family,\\
filial piety begins.\\
When the country falls into chaos,\\
patriotism is born.\\

\chapter*{19}
Throw away holiness and wisdom,\\
and people will be a hundred times happier.\\





\end{document}